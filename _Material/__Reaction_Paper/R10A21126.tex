\documentclass[]{article}

% 快點寫功課!!!
% \documentclass[]{article}
\makeatletter\if@twocolumn\PassOptionsToPackage{switch}{lineno}\else\fi\makeatother

  
\usepackage{amsmath,amsfonts,amsbsy,amssymb,tabulary,graphicx,times,caption,fancyhdr}
\usepackage[utf8]{inputenc}
\usepackage[paperheight=10in,paperwidth=6.5in,margin=2cm,headsep=.5cm,top=2.5cm,headheight=1cm]{geometry}
\renewenvironment{abstract} {\vspace*{-1pc}\trivlist\item[]\leftskip\oupIndent\hrulefill\par\vskip4pt\noindent\textbf{\abstractname}\mbox{\null}\\\relax}{\par\noindent\hrulefill\endtrivlist} 
\linespread{1.13} \date{}
\captionsetup[figure]{labelfont=sc,skip=1.4pt,aboveskip=1pc}
\captionsetup[table]{labelfont=sc,skip=1.4pt,labelsep=newline}



\makeatletter\def\oupIndent{1pt}
\def\author#1{\gdef\@author{\hskip-\dimexpr(\tabcolsep)\hskip\oupIndent\parbox{\dimexpr\textwidth-\oupIndent}{\centering\bfseries#1}}}
\def\title#1{\gdef\@title{\centering\bfseries\ifx\@articleType\@empty\else\@articleType\\\fi#1}}
\let\@articleType\@empty \def\articletype#1{\gdef\@articleType{{\normalfont\itshape#1}}}
\fancypagestyle{headings}{\fancyhf{}\fancyhead[C]{\RunningHead}\fancyhead[R]{\thepage}}\pagestyle{headings}
\makeatother

  


\tolerance=5000
%%%%%%%%%%%%%%%%%%%%%%%%%%%%%%%%%%%%%%%%%%%%%%%%%%%%%%%%%%%%%%%%%%%%%%%%%%
% Following additional macros are required to function some 
% functions which are not available in the class used.
%%%%%%%%%%%%%%%%%%%%%%%%%%%%%%%%%%%%%%%%%%%%%%%%%%%%%%%%%%%%%%%%%%%%%%%%%%
\usepackage{
  url,
multirow,morefloats,floatflt,cancel,tfrupee}
% \usepackage[hyphens]{url}
\usepackage{hyperref} 



\makeatletter


\AtBeginDocument{\@ifpackageloaded{textcomp}{}{\usepackage{textcomp}}}
\makeatother
\usepackage{colortbl}
\usepackage{xcolor}
\usepackage{pifont}
\usepackage[nointegrals]{wasysym}
\usepackage{enumitem}

\urlstyle{rm}
\makeatletter

%%%For Table column width calculation.
\def\mcWidth#1{\csname TY@F#1\endcsname+\tabcolsep}

%%Hacking center and right align for table
\def\cAlignHack{\rightskip\@flushglue\leftskip\@flushglue\parindent\z@\parfillskip\z@skip}
\def\rAlignHack{\rightskip\z@skip\leftskip\@flushglue \parindent\z@\parfillskip\z@skip}

%Etal definition in references
\@ifundefined{etal}{\def\etal{\textit{et~al}}}{}


%\if@twocolumn\usepackage{dblfloatfix}\fi
\usepackage{ifxetex}
\ifxetex\else\if@twocolumn\@ifpackageloaded{stfloats}{}{\usepackage{dblfloatfix}}\fi\fi

\AtBeginDocument{
\expandafter\ifx\csname eqalign\endcsname\relax
\def\eqalign#1{\null\vcenter{\def\\{\cr}\openup\jot\m@th
  \ialign{\strut$\displaystyle{##}$\hfil&$\displaystyle{{}##}$\hfil
      \crcr#1\crcr}}\,}
\fi
}

%For fixing hardfail when unicode letters appear inside table with endfloat
\AtBeginDocument{%
  \@ifpackageloaded{endfloat}%
   {\renewcommand\efloat@iwrite[1]{\immediate\expandafter\protected@write\csname efloat@post#1\endcsname{}}}{\newif\ifefloat@tables}%
}%

\def\BreakURLText#1{\@tfor\brk@tempa:=#1\do{\brk@tempa\hskip0pt}}
\let\lt=<
\let\gt=>
\def\processVert{\ifmmode|\else\textbar\fi}
\let\processvert\processVert

\@ifundefined{subparagraph}{
\def\subparagraph{\@startsection{paragraph}{5}{2\parindent}{0ex plus 0.1ex minus 0.1ex}%
{0ex}{\normalfont\small\itshape}}%
}{}

% These are now gobbled, so won't appear in the PDF.
\newcommand\role[1]{\unskip}
\newcommand\aucollab[1]{\unskip}
  
\@ifundefined{tsGraphicsScaleX}{\gdef\tsGraphicsScaleX{1}}{}
\@ifundefined{tsGraphicsScaleY}{\gdef\tsGraphicsScaleY{.9}}{}
% To automatically resize figures to fit inside the text area
\def\checkGraphicsWidth{\ifdim\Gin@nat@width>\linewidth
	\tsGraphicsScaleX\linewidth\else\Gin@nat@width\fi}

\def\checkGraphicsHeight{\ifdim\Gin@nat@height>.9\textheight
	\tsGraphicsScaleY\textheight\else\Gin@nat@height\fi}

\def\fixFloatSize#1{}%\@ifundefined{processdelayedfloats}{\setbox0=\hbox{\includegraphics{#1}}\ifnum\wd0<\columnwidth\relax\renewenvironment{figure*}{\begin{figure}}{\end{figure}}\fi}{}}
% \let\ts@includegraphics\includegraphics

% \def\inlinegraphic[#1]#2{{\edef\@tempa{#1}\edef\baseline@shift{\ifx\@tempa\@empty0\else#1\fi}\edef\tempZ{\the\numexpr(\numexpr(\baseline@shift*\f@size/100))}\protect\raisebox{\tempZ pt}{\ts@includegraphics{#2}}}}

% \renewcommand{\includegraphics}[1]{\ts@includegraphics[width=\checkGraphicsWidth]{#1}}
% \AtBeginDocument{\def\includegraphics{\@ifnextchar{\ts@includegraphics}{\ts@includegraphics[width=\checkGraphicsWidth,height=\checkGraphicsHeight,keepaspectratio]}}}

\DeclareMathAlphabet{\mathpzc}{OT1}{pzc}{m}{it}

\def\URL#1#2{\@ifundefined{href}{#2}{\href{#1}{#2}}}


\PassOptionsToPackage{hyphens,spaces,obeyspaces}{url}

% \usepackage{hyphens,spaces,obeyspaces}{url}
\def\UrlBreaks{\do\/\do-}
\usepackage{xurl}
\usepackage{hyperref}
\hypersetup{breaklinks=true}
% \makeatletter

%For url break

\def\UrlOrds{\do\*\do\-\do\~\do\'\do\"\do\-\do\/}%
% \g@addto@macro{\UrlBreaks}{\UrlOrds}


\def\UrlAlphabet{%
      \do\a\do\b\do\c\do\d\do\e\do\f\do\g\do\h\do\i\do\j%
      \do\k\do\l\do\m\do\n\do\o\do\p\do\q\do\r\do\s\do\t%
      \do\u\do\v\do\w\do\x\do\y\do\z\do\A\do\B\do\C\do\D%
      \do\E\do\F\do\G\do\H\do\I\do\J\do\K\do\L\do\M\do\N%
      \do\O\do\P\do\Q\do\R\do\S\do\T\do\U\do\V\do\W\do\X%
      \do\Y\do\Z}
\def\UrlDigits{\do\1\do\2\do\3\do\4\do\5\do\6\do\7\do\8\do\9\do\0}
\g@addto@macro{\UrlBreaks}{\UrlOrds}
\g@addto@macro{\UrlBreaks}{\UrlAlphabet}
\g@addto@macro{\UrlBreaks}{\UrlDigits}
% \makeatother


\edef\fntEncoding{\f@encoding}
\def\EUoneEnc{EU1}
\makeatother
\def\floatpagefraction{0.8} 
\def\dblfloatpagefraction{0.8}
\def\style#1#2{#2}
\def\xxxguillemotleft{\fontencoding{T1}\selectfont\guillemotleft}
\def\xxxguillemotright{\fontencoding{T1}\selectfont\guillemotright}

\newif\ifmultipleabstract\multipleabstractfalse%
\newenvironment{typesetAbstractGroup}{}{}%

%%%%%%%%%%%%%%%%%%%%%%%%%%%%%%%%%%%%%%%%%%%%%%%%%%%%%%%%%%%%%%%%%%%%%%%%%%



% \usepackage[numbers,sort&compress]{natbib}

\usepackage[sorting=none, citestyle=verbose-inote,backref=true,ibidtracker=context,mincrossrefs=99,backend=biber, 
% url = true,
% urlseen = false,
% url = false,
doi = false, isbn=false,]{biblatex}


\AtEveryCitekey{\clearfield{url}}
\AtEveryCitekey{\clearfield{urlyear}}
%去除 visited on...
% https://tex.stackexchange.com/questions/278890/biblatex-suppressing-urldate-does-not-work-clearfield
% urldate is split up into urlyear, urlmonth and urlday. Removing urlyear solves the problem.


% \AtEveryBibitem{
%     \clearfield{urlyear}
%     \clearfield{urlmonth}
% }


% \usepackage[style=verbose]{biblatex}



% \usepackage[style=verbose-inote,backref=true,ibidtracker=context]{biblatex}

% \usepackage[style=verbose-ibid]{biblatex}



\usepackage{xurl}

% \usepackage[style=verbose-ibid]{biblatex}
% \AtEveryCitekey{\clearfield{url}}
% \AtEveryCitekey{\clearfield{howpublished}} 


% If you want to break on URL numbers
\setcounter{biburlnumpenalty}{9000}
% If you want to break on URL lower case letters
\setcounter{biburllcpenalty}{9000}
% If you want to break on URL UPPER CASE letters
\setcounter{biburlucpenalty}{9000}

\biburlnumskip=0mu plus 1mu\relax
\biburlucskip=0mu plus 1mu\relax
\biburllcskip=0mu plus 1mu\relax

% Document
% https://www.emse.fr/~picard/files/biblatex.pdf
% Cheatsheet
% https://ftp.ntou.edu.tw/ctan/info/biblatex-cheatsheet/biblatex-cheatsheet.pdf

\addbibresource{R10A21126.bib}

\usepackage{lipsum} 
\usepackage[version=4]{mhchem}

\usepackage[T1]{fontenc}

%%%%%%%%%%%%%%%%%%%%%%%%%%%%%%%%%%%%%%%%%%
% Feature enabled:
%full-reference: true
%toc: yes
%%%%%%%%%%%%%%%%%%%%%%%%%%%%%%%%%%%%%%%%%%
\makeatletter\@ifundefined{tableofcontents}{\usepackage{typeset-custom-toc}}{}\makeatother
\usepackage{etoolbox}

% \usepackage{etoolbox}
% \patchcmd{\tableofcontents}{\@starttoc}{\vspace{-1cm}\@starttoc}{}{}

% defines the paragraph spacing
\setlength{\parskip}{0.5em}


% https://tex.stackexchange.com/questions/95838/how-to-write-a-perfect-equation-parameters-description

% \newlength{\conditionwd}
% \newenvironment{conditions}[1][where:]
%   {%
%    #1\tabularx{\textwidth-\widthof{#1}}[t]{
%      >{$}l<{$} @{${}={}$} X@{}
%    }%
%   }
%   {\endtabularx\\[\belowdisplayskip]}

\newenvironment{conditions}[1][where:]
{#1 \,\\ \begin{tabular}[t]{>{$}r<{$} @{${}={}$} p{0.91\linewidth}}}
{\end{tabular}\\[\belowdisplayskip]}


\usepackage{multirow}
\usepackage{booktabs}
\usepackage{hhline}


\usepackage{mhchem}
% \usepackage{graphicx}

% \graphicspath{ {images/} }

\begin{document}


\title{Reaction Paper}
\author{\textbf{\fontsize{14pt}{16.4pt}\selectfont{YIFAN WANG}}~\\\normalsize\normalfont {College of  Law \unskip, National Taiwan University}~\\{\normalsize\normalfont  E-mail: R10A21126@ntu.edu.tw}}
\def\RunningHead{{Reaction Paper}}

\maketitle 


% \begin{abstract}

%   This review introduces the paper "Co-benefits of carbon and pollution control policies on air quality and health till 2030 in China" by Jinzhao Yang et al. published in Environment International in July 2021.
%   The study used an integrated framework combining an energy-economic model, an air quality model, and concentration-response models to develop serval combined scenarios for 2030. 
%   Based on the simulation, the study evaluated China’s health benefits of carbon and air pollution control policies. Furthermore, a perspective on future environmental regulations was presented.

%   % This study examines the correlation between air pollution and climate change by analyzing existing literature. It also tries to sort out the interaction and co-benefit of regulations targeting the two environmental issues. Furthermore, the paper presents a perspective on future environmental regulation policies.


% \end{abstract}

\def\keywordstitle{Keywords}

\smallskip\noindent\textbf{Keywords: }{Judicial Independence with Chinese Characteristics} 


% \clearpage
\setcounter{tocdepth}{1}

% \tableofcontents


% \pagebreak

% % Introduction
% \section{Introduction}
% \label{sec:intro}

% \subsection{Background}
According to Tom Ginsburg, judicial control is one of the three categories of mechanisms that rulers can choose from to reduce agency costs, along with ideology and hierarchy. In the context of China, the article argues that some level of third-party monitoring is permissible due to the lack of complete independence of the courts from Communist Party influence in administrative affairs.

Judicial independence is considered to be one of the core elements of the rule of law. It ensures that judicial institutions exercise their adjudicative powers free from external interference or influence. 
Since the judiciary is designed to resolve disputes, neutrality and independence are its inherent nature. In China, however, the tradition of unity of law and politics is deeply rooted. 
The ancient Chinese judiciary lacked the characteristics of the modern judiciary that we are familiar with. Instead, it primarily exhibited attributes associated with ancient politics, such as administrative subordination, administrative control, administrative obedience, and administrative decision-making. Consequently, the ancient Chinese judiciary functioned as a subordinate tool under autocratic rule, serving as a support mechanism for administrative agencies.

Due to its inherent connection with state power, judiciary has long been under the control of state authority throughout ancient China. The judiciary in Western culture has theoretically achieved liberation from the constraints of state power after enduring centuries of struggle. However, to this day, it still struggles to completely escape the shadow of power control, in the context of China. 

As I know, from the current legislative point of view, in today's China, power interference in the administration of justice is not legitimate. There are many restrictive provisions regarding power interference in the judiciary, such as the clear stipulation in the Chinese Constitution regarding the independent adjudicative power of judicial institutions, nevertheless, the repeated exhortations from party and government organs actually reflect the persistent nature of power interference in the judiciary.

The arguement of Tom Ginsburg’s article actually raises a question for me, does the Chinese authority really want to promote the independence of judiciary, at the cost of losing one of the three mechanisms to reduce agency costs? How does the authority strike a balance between the benefits of monitoring mechanisms and the pursuit of judicial independence? The answer lies in China's unique definition of judicial independence, which differs from the "Western-version." China emphasizes that judicial independence should be carried out under the leadership of the Party, and that courts and judges should firmly support the Party's leadership and policies to ensure that judicial work meets national interests and development needs. Indeed, China has a distinct approach to the concept of judicial independence that deviates from universal values.  As part of judicial reform and development, Chinese officials are dedicated to enhancing judicial fairness and efficiency to ensure that the judicial system effectively serves the interests of the state and society. It is important to recognize that China's perspective on judicial independence is shaped by its unique political and cultural context.


\pagebreak


\sloppy

\printbibliography
  % \end{refcontext}
  % \printbibheading
  % \printbibliography[type=book,heading=subbibliography,title={Book Sources}]
  % \printbibliography[nottype=book,heading=subbibliography,title={Other Sources}]

\end{document}
