\documentclass[]{article}

\input{preamble.tex}

% \graphicspath{ {images/} }

\begin{document}


\title{\vspace{-3em}
Civic Constitutionalism and Tax Law\\
\large An Exploration of Civic Engagement in Taiwan}
\author{\textbf{\fontsize{14pt}{16.4pt}\selectfont{YIFAN WANG}}~\\\normalsize\normalfont {College of  Law \unskip, National Taiwan University}~\\{\normalsize\normalfont  E-mail: R10A21126@ntu.edu.tw}}
\def\RunningHead{{Civic Constitutionalism and Tax Law}}

\maketitle 

\vspace{-1em}
\begin{abstract}
  \vspace{-1em}

  This paper explores the relationship between taxes and civic constitutionalism in modern welfare states. The author discusses the importance of tax consciousness and compliance, as well as the role of the rule of law in the fiscal domain. The paper also examines the practice of civic constitutionalism in Taiwan's tax law, including the constitutional interpretation and the actions of civic organizations. The author concludes by reflecting on the importance of civic-centered approaches to fiscal law.

\end{abstract}

\def\keywordstitle{Keywords}

\smallskip\noindent\textbf{Keywords: }{Civic Constitutionalism, Tax Law, Tax Consciousness, Civic Engagement
% , Taiwan
} 

%  Fiscal Domain
% Taxes, Modern Welfare States, Civic Constitutionalism,  Tax Compliance, Rule of Law, Fiscal Domain, Taiwan, Tax Law, Constitutional Interpretation, Civic Organizations, Civic-centered Reflection, Fiscal Law.


% \clearpage
\setcounter{tocdepth}{2}

\tableofcontents


\pagebreak

% Introduction
\section{Introduction}
\label{sec:intro}



\subsection{Motivation}
\label{sec:motivation}

In the context of China, when you see the word "artist" and "tax" together, you are probably reading about the cases where celebrities or influencers evade or avoid income taxes illegally. In severe cases, individuals and the companies involved are blacklisted by the regulatory authorities.\footcite{2021}
However, in Taiwan, people who concern with tax law or public law in general, definitely know about the Constitutional Court Interpretation No. 745 that arose partly from the tax cases involving well-known artists: Lin Chi-Ling and Novia Lin. Declaring the tax law (article 17, paragraph 1 of the Income Tax Code) at the time unconstitutional, the Interpretation promoted Taiwan's tax law to comply with the principle of equality.

Famous artists are commonly considered to be the groups that can achieve high income levels, thus have a higher tax burden capacity in the context of John Rawls' theory of justice. 
As above, the difference in the roles the group play in the field of tax law inspires this study. This paper wants to find out the relationship between the environment of civic constitutionalism and tax consciousness in modern welfare states.

\subsection{Background}


\subsubsection{The Idea of Civic Constitutionalism }


Civic constitutionalism is a concept that the maintenance and sustainability of constitutional democracy depend heavily on a robust understanding of civic duty and citizenship as well as an ambitious and deeply democratic project of civic education in the principles and commitments central to a constitutional way of life.\footcite[7]{finn2017other}
According to the civic constitutionalist perspective, the people have a central role in the formal process of amending the constitution, and in its ongoing maintenance, interpretation, evolution, and even termination. 
% As a matter of constitutional practice, the civic constitutionalist view would give the people a central role not only in formal amendment but also in constitutional maintenance, interpretation, evolution, and termination.
\footcite[at 382]{williams2004civic}
% Civic constitutionalism is a concept that refers to the active role of citizens in shaping and interpreting the constitution. It emphasizes the importance of civic engagement in the development and maintenance of constitutional democracy. 
This idea has been applied in various contexts, including Taiwan, where it has been used to describe the transformative constitutional changes that have taken place in recent decades.\footcite{Yeh2015}

% (https://academic.oup.com/icon/article/16/2/702/5036463)
% \footcite{Yeh_2016a}

Civic constitutionalism can be manifested through the following characteristics.\footcite[at 320]{Yeh2015} Firstly, civil society actively participates in the democratic process by engaging Congress and promoting representative democracy through constitutional reforms. This involvement allows civil society to contribute to the interpretation, understanding, and shaping of the constitution by exerting indirect or direct influence on their elected representatives.

Secondly, since the democratization process, civil society remains vigilant and engaged, even when there is a trusted Constitutional Court in place. On one hand, civil society serves as a driving force for judicial review, advocating for the protection of rights and government accountability by bringing relevant issues to the courts for resolution. This process contributes to the strengthening and consolidation of constitutional principles and values. On the other hand, while the court interprets and applies the constitution, it remains attentive to public sentiment and opinions regarding constitutional adjudication matters.

Thirdly, engaged citizens and civic groups play a crucial role in initiating, deliberating, and steering constitutional reform agendas, despite the institutionalized processes for constitutional amendments. Their active participation enriches our understanding of constitutions and how they function within post-democratized societies.



\subsubsection{The Fiscal System of the Modern Welfare State}

The fiscal system of the modern welfare state is characterized by a combination of taxation and social security contributions to fund public expenditures on social welfare programs such as education, healthcare, social security payments, and housing.\footcite{2023a}
Taxation serves as a crucial source of government revenue in the modern welfare state.
Even anarchists may consider taxation as a necessary source of public funds for essential public goods and services in the short term.\footcite[30]{chomsky2013anarchism}



% According to the book “Global Taxation: How Modern Taxes Conquered the World” by Philipp Genschel and Laura Seelkopf, 
The introduction of modern taxes took the state’s fiscal capacity to a completely new level, expanding its revenue potential beyond anything previously conceivable and furnishing it with a rich toolkit of redistributive instruments.
\footcite[1]{Genschel2021} For example, progressive taxation is often used to redistribute wealth and reduce income inequality.\footcite{Kato2003}
% This enabled the emergence of big government as we know it today, including the modern welfare state. 
Without a robust tax system, the modern welfare state would not have the necessary funds to provide social welfare programs and other public goods and services to its citizens. In short, a robust tax system is necessary for the modern welfare state to function effectively. 


\section{Tax Consciousness and Civic Constitutionalism}


In this paper, taxation is considered to be a field of civic engagement, where people actively participate in the democratic process by contributing to the functioning of the government and public services. 
This section examines the importance of tax consciousness from the perspective of civic constitutionalism. "Tax consciousness", or "tax awareness",\footcite[at 237]{Buehler1940} refers to individuals' awareness and understanding of taxation. This includes taxpayers understanding their tax obligations and recognizing the importance of taxation to society and their contributions.

\subsection{Tax Compliance}


One of the most tangible aspects of tax consciousness is tax compliance, which refers to individuals' adherence to tax laws and regulations by voluntarily fulfilling their tax obligations. It involves timely and accurate reporting of income, assets, and relevant information, as well as the payment of taxes as required by law. Tax compliance is crucial for the proper functioning of the tax system, as it relies on taxpayers’ willingness to comply with tax laws and fulfill their obligations. 

\subsection{Fundamention of Tax Compliance}

Some scholars in the natural or social sciences may simplify tax compliance into a simple network model, focusing only on interpersonal influences while overlooking factors related to the specific tax system or the rule of law.\footcite{DiGioacchino2020} However, this paper argues that tax compliance should be regarded as a product of civic constitutionalism. Tax compliance, as the outward manifestation of tax consciousness, depends significantly on the level of civic engagement.

% As the fundamention of tax compliance,

Tax consciousness exhibits various dimensions.
It involves not only the awareness of tax obligations and responsibilities but also the attention given to the fairness and equality of the tax system. Taxpayers' concern for the fairness and equality of tax laws reflects their expectations of the rationality and justice of the fiscal system. This can be separated into two sides, the tax burden and the utilization of public funds. Taxpayers are expected to be attentive to whether taxes are distributed fairly and whether there are preferential or unfair tax treatments. utilization and allocation of tax funds. Besides, taxpayers are also concerned about how the government utilizes tax revenues, ensuring alignment with the public interest, and expressing a desire for fair and reasonable distribution of funds.

Under the rule of law of fiscal taxation, a state with sufficient civic participation is an important and necessary prerequisite for tax compliance.
% Secondly, tax consciousness encompasses knowledge and understanding of the tax system and tax laws. This includes awareness of tax rules, tax rates, and tax procedures, as well as interest in tax policies and reforms.
% Moreover, tax consciousness includes concern for tax fairness and equality. 
% Additionally, tax consciousness extends to the utilization and allocation of tax funds. 
% Lastly, tax consciousness reflects the importance of tax compliance. Taxpayers emphasize the significance of adhering to tax laws and reporting procedures to ensure compliance and achieve lawful tax obligations.
% In summary, tax consciousness is a multifaceted concept encompassing awareness of tax obligations and responsibilities, understanding of the tax system, concern for tax fairness and equality, and attention to the utilization and allocation of tax funds.
% These dimensions are mutually reinforcing. 
Please recall Section 12 of the Magna Carta, which states that no scutage or aid (tax) shall be imposed without the consent of the common council of the realm. This means that the monarch could not levy taxes without the barons' approval, establishing the principle of obtaining consent for taxation and setting the stage for the development of civic constitutionalism.
The concept of "no taxation without representation" became a fundamental principle of civic constitutionalism. 
It asserted that individuals should have a say in the imposition of taxes and the allocation of public resources through their elected representatives. This idea was later echoed in the American Revolution and the drafting of the United States Constitution, further solidifying the relationship between civic constitutionalism and taxation.\footcite{2023b}

In summary, civic engagement has long been an important factor in shaping the fiscal system. In modern welfare states, citizens' active participation and engagement in the democratic process can influence the design of taxation policies, the allocation of public funds, and the priorities of government spending. A trusted fiscal system with enough civic engagement serves as the foundation of tax compliance.

% the Magna Carta  It emphasized the importance of civic engagement, limiting arbitrary taxation, and ensuring that taxation policies align with the principles of justice and fairness. These principles have had a lasting impact on constitutional development and governance around the world.



% For instance, **the Magna Carta's section 12** stipulates that "No scutage or assistance may be charged in our realm without its universal assent," 

% The Magna Carta played a crucial role in shaping this tradition of civic constitutionalism by emphasizing the importance of consent and consultation in matters of taxation.


% [No taxation without representation - Wikipedia](https://en.wikipedia.org/wiki/No_taxation_without_representation)


% In summary, the Magna Carta  It emphasized the importance of civic engagement, limiting arbitrary taxation, and ensuring that taxation policies align with the principles of justice and fairness. These principles have had a lasting impact on constitutional development and governance around the world.

\section{Practice of Civic Constitutionalism in Taiwan}

In Taiwan, the constitutional experience of the past few decades has been filled with violations and reconstruction of rights. Since democratization, the fundamental rights enshrined in the Constitution have been undergoing a process of constant argumentation. This dynamic development of rights has been primarily following three main paths that intertwine: civil society, the judicial path, and constitutional reform, corresponding to the main characteristics of civic constitutionalism mentioned earlier.

In the field of taxation, the practices of argumentation could also follow the above patterns. In this section, examples are illustrated to demonstrate the practice of civic constitutionalism in Taiwan in the field of tax law, following two of the main paths above. 



% The examples 
% 對應前述的cc的特徵

% Practice of Civic Constitutionalism in the Field of Fiscal Tax Law in Taiwan

% 在台灣,近幾十年來的憲政經驗,充滿著權利的侵害與重建。民主化之後,憲法上基本權利更是經過了一段反復論證的過程。這段權利論證的動態發展,主要是遵循了三條主要的路徑而交織形成的,分別是市民社會、司法路徑,與憲政改革。


\subsection{Constitutional Review}


Constitutional Court Interpretation No. 745 was caused by Litigations filed by a tax law professor Chen Ching-Shiou
and a model Novia Lin.
The two taxpayers are considered to be earners of salary income and are not allowed to deduct the full amount of their expenses according to the tax law at that time. This leads to a greater amount of tax burden, compared to the taxpayers who are considered to be practitioners, which may be against the principle of equality from the Constitution.
Therefore, they asserted that taxpayers categorized to be earners of salary income should also be allowed to deduct the full amount of their expenses.\footnote{Professor Chen also asserted that the hourly pay for his teaching at a university in 2008 should be categorized as income earned by a practitioner, rather than salary income.}
% and he objected to the administrative assessment made by Taipei City Tax Bureau, Ministry of Finance that categorized his earnings as salary income. After losing the case, he filed an administrative appeal. 
% After the appeal was also dismissed, he initiated administrative court proceedings. When his case was dismissed by judgment Jian Zi No. 236 (2011) of the Taipei High Administrative Court, the petitioner appealed, but the appeal was dismissed by ruling Cai Zi No. 196 (2012) of the Supreme Administrative Court for failing to specify how the appeal judgement was inconsistent with the law1. Accordingly, judgment Jian Zi No. 236, rendered by the Taipei High Administrative Court in 2011, should be the final court judgment that brought about this constitutional interpretation1. 
After suspending the litigation procedure, the Judge applied for a constitutional interpretation.
%  pursuant to Judicial Yuan interpretation No. 371, No. 572 and No. 5901.


% Constitutional Court Interpretation No. 745 of the Republic of China (ROC) was issued on February 8, 2017. It addressed the issue of whether it is unconstitutional to disallow earners of salary income to deduct the full amount of their expenses1. 
The interpretation found that salary earners are allowed to deduct from their personal incomes only a fixed amount of the Special Deduction Amount for Salary Income. When the necessary expenses exceed the statutory Deduction Amount per year, salary earners are not allowed to deduct necessary expenses either by enumeration or other methods, which is inconsistent with the right to equal treatment under Article 7 of the Constitution. The relevant authorities were instructed to review and amend the Income Tax Act and relevant regulations in line with this Interpretation within two years from the announcement of this Interpretation\footcite{zotero-175}.


This Interpretation serves as a perfect and reassuring example of effective civic constitutionalism in action, in the field of tax law. In response to taxpayers' persistent claim for rights, the constitutional court's decision helps to shape and reform the tax law in accordance with constitutional principles. This Interpretation signals the Constitutional Court's shift from the formal review of tax legalism to the substantive review of the rule of law. It has shaped and enriched the connotation of the constitution, embodied the argument of citizens' rights in the judicial channel, and realized the citizens' imagination of the constitutional order.

Although Taiwan is now a relatively mature constitutional society with a rule of law, there are still instances where the preceding legal system is not conducive to the fundamental human rights of the Constitution. The indispensability of citizen participation lies in its capability to identify and rectify these scattered flaws throughout the legal system.


% 形塑、豐富了憲法的内涵,體現了市民在司法途徑上的權利論證,實現公民對憲政秩序的想象。

% Civic constitutionalism refers to a situation where the strength and resilience of civic engagement pushes towards democratic transition and constitutional reform2. In Taiwan, civic constitutionalism has been achieved through constitutional interpretation, which essentially becomes constitution-making2. The small-c constitution and unwritten constitution may be more pivotal than the capital-C Constitution in understanding the constitutional contours of Taiwan2. In this context, Constitutional Court Interpretation No. 745 can be seen as an 


% As mentioned in the previous section, the

% The ruling addressed the issue of whether it is unconstitutional to disallow earners of salary income to deduct the full amount of their expenses. The court found that the relevant provisions of the Income Tax Act were inconsistent with the right to equal treatment under Article 7 of the Constitution, and instructed the relevant authorities to review and amend the Income Tax Act and relevant regulations in accordance with this Interpretation within two years from the announcement of this Interpretation.
% \footcite{zotero-175}


% 從租稅法律主義的形式審查,到財稅法治的實質審查

\subsection{Contining Actions of Civic Organizations}

From time to time, people can encounter a group of protestors with yellow vests and flags in the main streets of Taipei, conducting tax-related speeches and demonstrations. They are associated to Tai Ji Men Qigong Academy, a spiritual and cultural organization in Taiwan. 
The protests are raised from the protracted tax disputes they have, called Tai Ji Men Tax Case.


% The Tai Ji Men Tax Case refers to a legal dispute in Taiwan involving the Tai Ji Men Qigong Academy, a spiritual and cultural organization, and its tax obligations. The case has garnered significant attention and sparked social activism related to tax fairness and religious freedom.


In 1992, the Taiwanese government initiated an investigation into the tax status of the Tai Ji Men Qigong Academy. The tax administration claimed that the academy owed substantial taxes and imposed heavy fines and penalties. The tax administration considers this organization to be a profit-making business in essence. Tai Ji Men disputed these claims, arguing that it qualified for tax exemptions as a non-profit religious organization. The supporters advocate that the money is a gift of thanks to the teacher (master), not tuition fees.
The case went through a protracted legal battle, spanning more than two decades. Tai Ji Men sought to defend its tax-exempt status based on its religious nature and the social contributions it made through its cultural and charitable activities.



\subsubsection{The Conflict between High Tax Burden and Freedom of Belief}

The Tai Ji Men Qigong Academy, founded in 1966, promotes the practice of Qigong, a traditional Chinese discipline focused on cultivating physical and mental well-being. The organization's leader, Dr. Hong, has been actively involved in promoting Qigong internationally.
Supporters of Tai Ji Men rallied behind the organization, emphasizing the importance of religious freedom and fair treatment in tax matters. They highlight the potential implications of the case on religious organizations and their ability to fulfill their social and cultural missions.

Since the dispute, various social groups and activists have been expressing their support for Tai Ji Men and participated in public demonstrations, petitions, and awareness campaigns. They argue that the case is not just about Tai Ji Men's tax obligations but also about upholding the principles of equality, religious freedom, and fair treatment under the law.
These social actions are aimed to raise public awareness about the Tai Ji Men Tax Case, garner support from the general population, and put pressure on the government to reconsider its stance. 

% Supporters of Tai Ji Men .

The Tai Ji Men Tax Case has prompted broader discussions and debates about tax fairness, religious freedom, and the appropriate treatment of non-profit organizations in Taiwan. It has led to calls for comprehensive tax reform and clearer guidelines regarding the tax status of religious and non-profit organizations.

These protesting activities are not only centered around the tax obligations of the Tai Ji Men Qigong Academy but has also ignited social activism related to tax fairness and religious freedom. Freedom of belief encompasses the freedom to practice, observe, and express one's religious convictions without undue interference from the state. When taxation policies unduly burden religious practices or force individuals to compromise their religious beliefs, it can be seen as a violation of their freedom of belief. The case has catalyzed discussions on tax reform and the treatment of non-profit organizations, contributing to a broader dialogue on these important social issues. \footcite{Jacobsen2020}


% The conflict between high tax burden and freedom of belief arises when the imposition of excessive taxes or financial obligations hinders or infringes upon an individual's or a religious organization's freedom to practice and express their beliefs. This clash often occurs when tax policies and regulations place a significant financial burden on religious institutions or their members, potentially impeding their ability to exercise their religious rights.

% High tax burden can create challenges for religious organizations, as they rely on financial resources to carry out their religious activities, maintain places of worship, and support their communities. Excessive taxation can restrict their ability to allocate funds for religious practices, charitable endeavors, and the preservation of cultural heritage.

% Furthermore, individuals who strongly adhere to their faith may find it burdensome to comply with tax requirements that conflict with their religious beliefs. For example, certain taxes may be perceived as contradictory to religious principles, leading to moral dilemmas or conscientious objections.



% Addressing the conflict between high tax burden and freedom of belief requires striking a balance between fiscal responsibilities and the protection of fundamental rights. Governments should consider the unique circumstances and needs of religious organizations, ensuring that tax policies do not unreasonably impede their ability to exercise their religious freedoms.

% Moreover, individuals should have the opportunity to seek exemptions or accommodations when tax requirements directly conflict with their religious beliefs. This can involve establishing clear guidelines and procedures for individuals and organizations to assert their rights and demonstrate the sincerity of their religious objections.

% By recognizing and respecting the importance of both fiscal responsibilities and freedom of belief, societies can work towards reconciling the conflict and ensuring that individuals and religious organizations can exercise their religious rights without undue financial burdens or infringements on their beliefs.

% The involved group claims that these related civic actions are not only tax-related protests, but also 
% religious movements. They think that the tax law is used as a tool for discrimination against spiritual minorities
% % https://www.cesnur.org/2020/tai-ji-men-case-and-beyond.htm
% .

\subsubsection{Government's Attitude of Inclusiveness towards Civil Society}

In such cases, conflicts arise between the government's tax policies and the fundamental right to freedom of belief. 
The Taiwan government has demonstrated an attitude of inclusiveness towards civil society, recognizing the importance of active civic engagement and the valuable role that civil society organizations play in democratic governance. This shows the great progress of Taiwan's rule of law society since democratization.

This example also demonstrates that the rights of taxation also interact with other constitutional rights, such as freedom of belief, freedom of expression, the right to assemble and march, and the right of association. This shows that fundamental rights are interconnected and indivisible, meaning that they are interdependent and mutually reinforcing. Each right contributes to the realization and protection of other rights, forming a cohesive framework for human dignity and equality. These rights function as a cohesive system, with each right strengthening and complementing the others. 
The peaceful implementation of these rights in this example is also a practice of citizen constitutionalism. 

As to whether these actions can ultimately promote tax law amendments, it still depends on the attitude of Congress, which represents public and professional opinions.

\section{Civic-centered Reflection upon Fiscal Law}

\subsection{Implications of Tax Disputes}

There's an interview with Noam Chomsky that talks about taxation.
\footcite{Citizensgateway2013}
% very good measure of how democratic a society is is to ask about the attitude towards taxes I mean if you had a free functioning democratic society April 15th you know when you pay our taxes that would be a day of celebration here we are getting together to fund the policies that we decided on that's great that's what we want to do here it's a day of mourning it's not a government of By and For the People you know that itself is a sign of the serious decline of the functioning of the democratic system
In the interview, he emphasizes the importance of a positive attitude toward taxes as an indicator of a healthy and functional democratic society, while expressing concern about the perceived decline in democratic functioning based on the current negative perception of Tax Day in the U.S. nowadays.

In Taiwan, the average fraction of the number of tax cases to all administrative appeals cases terminated by the Supreme Administrative Court is around 30\% for the last decade.\footcite{2022b} This, to a certain extent, means that there is still a lot of room for improvement in the current tax legal system. Reflection is required not just about the level of compliance, but more importantly the whole fiscal system, including the legislation. The practice of civic constitutionalism can have a substantial impact on the system. Of course, the impact is not limited to the field of taxation. As discussed in the sections above, the effects of civic constitutionalism could be comprehensive and far-reaching.
Remember the case of Moritz v. Commissioner in 1972, the true case Ruth Bader Ginsburg argues in the film \textit{On the Basis of Sex}? It is also remarkable in the context of civic constitutionalism because it was a landmark case in which the United States Court of Appeals for the Tenth Circuit held that discrimination based on gender constitutes a violation of the Equal Protection Clause of the United States Constitution. \footcite{2023} This case helped to establish the principle that laws that discriminate based on gender are unconstitutional and set a precedent for future cases challenging gender-based discrimination, advancing the cause of equal rights.

% Both this case and the Interpretation are extremely vital, but their implication is slightly different. In the background of the case of Moritz, gender egalitarianism was not as deeply rooted in the United States as it is now. The Court's ruling Declared New Era and promoted gender equality legislative reforms. Today, Taiwan is a relatively mature constitutional society governed by the rule of law.

% the case of Moritz v. Commissioner is important to both constitutional interpretation and civic constitutionalism. In terms of constitutional interpretation, the case was a landmark decision in which the United States Court of Appeals for the Tenth Circuit held that discrimination on the basis of gender constitutes a violation of the Equal Protection Clause of the United States Constitution. 

%  In terms of civic constitutionalism, this case helped to  and equal protection under the law for all citizens, regardless of their gender. It was also the first gender-discrimination suit Supreme Court Justice Ruth Bader Ginsburg argued in court. 



% Tax Legislation






% The speech highlights the significance of the attitude towards taxes as a measure of how democratic a society is. He argues that in a truly free and functional democratic society, paying taxes would be seen as a cause for celebration. It is suggested that citizens would willingly come together on Tax Day, to fund the policies that have been collectively decided upon. This positive perspective on taxes reflects a government that is truly "by and for the people."




% one of the comments below goes like this: 
% Tax day should be like Christmas Eve. Except you know you and the people you care about are going to have exactly what you and they want.

\subsection{Solution for Tax Consciousness}

For the instances where people lack awareness of tax obligations and seek to evade taxes through illegal transaction arrangements (of course not only in China), one of the main reasons may be the absence of civic engagement. People who have a negative attitude towards tax payment, in general, are not well-informed about the functions of public finance or do not identify with the fiscal system (due to insufficient supervisory review).

Therefore, the effective way to promote tax compliance, or tax morality, is to make sure that the fiscal-tax system is based on the will of the people, and the principles of rule of law are followed completely. To be specific, the revenue and the spending of the public finance funds should be sufficiently transparent under the supervision of the people. The design of the tax policy is fair and just. The government budget review mechanism follows proper procedures and is subject to checks and balances of power. This means that the process of reviewing and approving the government’s budget is subject to oversight and scrutiny by different branches of government, as well as by the public, to ensure that it is transparent, fair, and in the best interests of the people. 
When citizens perceive that their taxes are being used responsibly and for the betterment of society, they are more likely to willingly comply with their tax obligations.

% https://www.youtube.com/watch?v=gsUJGwZcVVU



\section{Conclusion}


% 稅法領域的公民憲政實踐,是台灣全面民主化的重要組成部分,也是公民參與法治建設的重要縮影


% The civic constitutional practice in the field of tax law is a crucial component of Taiwan's comprehensive democratization and serves as a significant reflection of citizen participation in the construction of the rule of law.
Taxpayers' rights are civil rights, that is, human rights.
The practice of civic constitutionalism in the field of tax law is a crucial component of Taiwan's comprehensive democratization and serves as a significant reflection of civic engagement in the construction of the rule of law. 
% Civic participation avenues are open and unhindered, allowing for active engagement and influence in shaping tax policies and ensuring fairness and equality. By embracing a civic-centered approach to fiscal law, Taiwan can continue to strengthen its democratic institutions and promote the principles of justice and fairness in its tax system. This essay highlights the importance of civic engagement in shaping tax policies and ensuring that they align with the principles of fairness, equality, and justice. Through active participation and engagement, citizens can help to shape a more just and equitable tax system that reflects their values and priorities.


This paper highlights the contributions of civic engagement to shaping tax policies and guaranteeing their alignment with the tenets of fairness, equality, and justice. 
By providing open and unhindered paths for civic engagement, Taiwan enables individuals to actively engage and exert influence in the formulation of tax policies, thus ensuring fairness and equality as much as possible. Embracing a civic-centered approach to fiscal legislation allows Taiwan to further strengthen its democratic institutions and uphold the principles of justice and fairness within its tax system. 





% Through active participation and engagement, citizens play a vital role in molding a tax system that is more equitable and reflective of their values and concerns.
This approach can serve as an inspiration for other countries, including China, to embrace a more democratic and inclusive approach to fiscal policy-making.
% As for the inspiration for China, this essay demonstrates how active citizen participation can play a crucial role in shaping tax policies that are fair, just, and equitable.
If one day, in China, people are empowered to engage with their government and advocate for their rights, unobstructedly, they can strive to ensure that their voices are heard and their interests are represented. Tax compliance then can be expected to improve as people know that the fiscal system is being optimized to efficiently utilize and generate public wealth, providing public services and assisting those in need.


By investigating these dynamics, the study intends to contribute to the existing literature on tax compliance and civic engagement, providing insights into how the environment of civic constitutionalism can foster a culture of tax consciousness and further promote the overall level of social justice. Ultimately, the findings of this research can inform policy discussions and initiatives aimed at enhancing tax compliance and promoting a more participatory and informed approach to fiscal matters.



The fiscal system is the cause, not the end. fiscal-induced civic engagement will bring about dynamic revolutions in pursuit of the ideal of democracy and rule of law, and realize social justice.

For further studies, the paper suggests that qualitative research or quantitative analysis is expected to be conducted focusing on the relationship of number of tax disputes, level of tax compliance, level of social welfare and level of democracy. Time-series and multinational materials are required for a comprehensive study. By conducting such research, a deeper understanding can be gained regarding the complex dynamics among the factors, shedding light on the factors that shape taxpayers' behaviors and the effectiveness of tax systems in modern societies.

% pursue the 


% 文章對於台灣的財稅法治采取樂觀的態度

% Civic participation avenues are open and unhindered.

% further study: quantification relationship between tax consciousness and the level of civic engagement in the context of civic constitutionalism.

\pagebreak


\sloppy

\printbibliography
  % \end{refcontext}
  % \printbibheading
  % \printbibliography[type=book,heading=subbibliography,title={Book Sources}]
  % \printbibliography[nottype=book,heading=subbibliography,title={Other Sources}]

\end{document}
